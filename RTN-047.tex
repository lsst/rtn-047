\documentclass[OPS,authoryear,toc]{lsstdoc}
\input{meta}

% Package imports go here.

% Local commands go here.

%If you want glossaries
%\input{aglossary.tex}
%\makeglossaries

\title{System Performance Management and Execution Plan}

% Optional subtitle
% \setDocSubtitle{A subtitle}

\author{%
Leanne Guy
}

\setDocRef{RTN-047}
\setDocUpstreamLocation{\url{https://github.com/lsst/rtn-047}}

\date{\vcsDate}

% Optional: name of the document's curator
% \setDocCurator{The Curator of this Document}

\setDocAbstract{%
This document describes the organization and management and execution of the System Performance department in Rubin Operations.  
It lays out the department goals and describes the department structure, roles and responsibilities as well as interactions with all other Rubin Operations departments. 
}

% Change history defined here.
% Order: oldest first.
% Fields: VERSION, DATE, DESCRIPTION, OWNER NAME.
% See LPM-51 for version number policy.
\setDocChangeRecord{%
  \addtohist{1}{YYYY-MM-DD}{Unreleased.}{Leanne Guy}
}


\begin{document}

% Create the title page.
\maketitle
% Frequently for a technote we do not want a title page  uncomment this to remove the title page and changelog.
% use \mkshorttitle to remove the extra pages

% ADD CONTENT HERE
% You can also use the \input command to include several content files.
%--------------------------- Introduction ----------------------------------------------------------------------- 
\section{Introduction}
\subsection{Purpose}

This document defines the mission, goals and objectives, organization and responsibilities of Vera C. Rubin Observatory System Performance Department.

\subsection{Mission Statement}

Rubin Observatory System Performance department is responsible for ensuring that the Legacy Survey of Space and Time (LSST) as a whole is proceeding with the efficiency and fidelity needed to achieve its science requirements at the end of the 10-year survey. 
This includes the Wide-Fast-Deep (WFD) survey and all Special Programs (deep drilling fields, Target of Opportunity (ToO) and mini/micro/nano surveys)


\subsection{Goals and Objectives}

{\emph Track  and optimise} the performance of the entire system including the observatory performance and survey progress with respect to the science objectives, and the evaluation of strategies for improving the survey strategy;

{\emph Provide QA and performance characterization analyses}  of the data products, understand and minimize systematics, gather user feedback – all as input  to the Data Release board to assess Rubin’s schedule to make annual data releases;

\emph{Enable the community to access and analyze the data} and publish results on the four LSST science pillars at an appropriate rate.

\emph{Oversight and implementation of Change and Process control} through a Change Control Board (CCB), and Risk and Opportunity Management, both chaired by the Lead Systems Engineer.

%------------------------Lifecycle of the Data--------------------------------------------------------------------------
\section{Approach -- Lifecycle of the Data} 

System Performance adopts a lifecycle of the data approach to its work. 


%-------------------------Processes and Tools-------------------------------------------------------------------------
\section{Processes and tools}


%------------------------Teams and Organization--------------------------------------------------------------------------
\section{Team and Organization}

System Performance comprises four functional teams. 

\input{validation}
\input{community}
\input{survey}
\input{syseng}
%-------------------------Roles and Responsibilities -------------------------------------------------------------------------
\section{Roles and Responsibilities}

% Extract a table from the WBS of all roles and responsibilities 

This section details the roles and responsibilities in System Performance. 
They come from the WBS 

%-------------------------Communications -------------------------------------------------------------------------
\section{Communications}

This section covers intra-department communications and 
Slack and email are used. 
High-level slack channels are used for prompt communications 

AHM once per year

%------------------------Product Tree --------------------------------------------------------------------------
\section{Product Tree }

This section details the System Performance products and product tree . 

Products include 1) Software, 2) Data Products and Documentation 3) Services, 


\appendix
% Include all the relevant bib files.
% https://lsst-texmf.lsst.io/lsstdoc.html#bibliographies
\section{References} \label{sec:bib}
\renewcommand{\refname}{} % Suppress default Bibliography section
\bibliography{local,lsst,lsst-dm,refs_ads,refs,books}

% Make sure lsst-texmf/bin/generateAcronyms.py is in your path
\section{Acronyms} \label{sec:acronyms}
\input{acronyms.tex}
% If you want glossary uncomment below -- comment out the two lines above
%\printglossaries





\end{document}
